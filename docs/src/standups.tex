%! Author = huibmeulenbelt
%! Date = 01/03/2025

% Preamble
\documentclass[11pt]{article}

% Packages
\usepackage{amsmath}
\usepackage{amsfonts}
\usepackage{amssymb}
\usepackage{graphicx}
\usepackage{titling}

\setlength{\droptitle}{-5cm}       % This is your set screw.
\graphicspath{ {./images/} }

% Document
\begin{document}
    \title{A MMArkov Chain}
    \maketitle

    \section*{Markov chain}
    A MMA fight can be represented as a Markov chain.
    Figure \ref{fig:chain} shows the relationships between the different states.
    Table \ref{tab:terminology} describes the different states and actions.

    \begin{figure}[h!]
      \centering
      \subfigure{\includegraphics[width=\linewidth]{chain.png}}
      \caption{A visual representation of the states and transitions}
      \label{fig:chain}
   \end{figure}

    \begin{table}[h!]
        \centering
       \begin{tabular}{@{}l l l}
           \hline
           \textbf{Variable} & \textbf{Definition} & \textbf{Category} \\
           \hline
           GRC & Ground control & state \\
           KDL & Knockdown landed & transition \\
           RVL & Reversal landed & transition \\
           ST  & Standing & state \\
           SUL & Standup landed & transition \\
           TDL & Takedown landed & transition \\ [1ex]
           \hline
       \end{tabular}
        \caption{A summary of the states and transitions}
        \label{tab:terminology}
    \end{table}

    \section*{Stand ups}
    All the variables of Table 1 are known, with the exception of $SUL$.
    The chain always starts in $ST$ and depending on the final state we can infer the following:

    \subsubsection*{Final state is \textbf{ST}}
    $GRC_{blue}$ and $GRC_{red}$ are entered and exited the same number of times.

    \begin{align*}
        SUL_{red} & = \max(0, KDL_{blue} + TDL_{blue} + RVL_{blue} - RVL_{red}) \\
        SUL_{blue} & = \max(0, KDL_{red} + TDL_{red} + RVL_{red} - RVL_{blue}) \\
    \end{align*}

    \subsubsection*{Final state is \textbf{GRC\_{blue}}}
    The number of entries of $GRC_{blue}$ is one greater than the number of exits.
    The number of entries and exits of $GRC_{red}$ is equal to each other.

    \begin{align*}
        SUL_{blue} & = \max(0, KDL_{red} + TDL_{red} + RVL_{red} - RVL_{blue} - 1) \\
        SUL_{red} & = \max(0, KDL_{blue} + TDL_{blue} + RVL_{blue} - RVL_{red}) \\
    \end{align*}

    \subsubsection*{Final state is \textbf{GRC\_{red}}}
    The number of entries and exits of $GRC_{blue}$ is equal to each other.
    The number of entries of $GRC_{red}$ is one greater than the number of exits.

    \begin{align*}
        SUL_{blue} & = \max(0, KDL_{red} + TDL_{red} + RVL_{red} - RVL_{blue}) \\
        SUL_{red} & = \max(0, KDL_{blue} + TDL_{blue} + RVL_{blue} - RVL_{red} - 1) \\
    \end{align*}

\end{document}