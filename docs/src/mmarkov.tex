%! Author = huibmeulenbelt
%! Date = 18/04/2025

% Preamble
\documentclass[11pt]{article}

% Packages
\usepackage{amsmath}

\pagestyle{empty}

% Document
\begin{document}
    \title{A MMArkov Chain}
    \date{}
    \maketitle

    \section*{Markov chain}
        A fight between fighters \textit{blue} and \textit{red} can be represented as a Markov chain.
        Although a real-world fight is very complex, we assume for this particular example that the following actions and states are possible:
    \begin{itemize}
        \item standing
        \item strike attempt blue
        \item strike landed blue
        \item strike attempt red
        \item strike landed red
        \item knockout blue
        \item knockout red
    \end{itemize}

    The transition matrix between the states and actions is constructed as follows:

    $$
    T = \begin{bmatrix}
    0.6 & 0.75 & 0.95 & 0.6 & 0.85 & 0 & 0 \\
    0.3 & 0 & 0 & 0 & 0 & 0 & 0 \\
    0 & 0.25 & 0 & 0 & 0 & 0 & 0 \\
    0.1 & 0 & 0 & 0 & 0 & 0 & 0 \\
    0 & 0 & 0 & 0.4 & 0 & 0 & 0 \\
    0 & 0 & 0.05 & 0 & 0 & 1 & 0 \\
    0 & 0 & 0 & 0 & 0.15 & 0 & 1
    \end{bmatrix}
    $$

    The first column represents the probabilities of the fight moving from "standing" to one of the other states.
    So, the probability of the fight staying in the "standing"-state is equal to $60\%$.
    The last two columns represents the "knockout blue"- and "knockout red"-states.
    These states are absorbing states; once the fight is in this state, it is impossible to move to another state.

    \break

    The vector $x_k = T^kx_0$ tell us the probability of being in any of the states after $k$ steps.

    $$
    x_1 = \begin{bmatrix}
    0.6 \\
    0.3 \\
    0 \\
    0.1 \\
    0 \\
    0 \\
    0 \\
    \end{bmatrix},
    x_2 = \begin{bmatrix}
    0.645 \\
    0.18 \\
    0.075 \\
    0.06 \\
    0.04 \\
    0 \\
    0 \\
    \end{bmatrix},
    x_{10} = \begin{bmatrix}
    0.6232 \\
    0.1882 \\
    0.0474 \\
    0.0627 \\
    0.0253 \\
    0.0205 \\
    0.0328 \\
    \end{bmatrix},
    x_{1000} = \begin{bmatrix}
    0.001 \\
    0.000 \\
    0.000 \\
    0.000 \\
    0.000 \\
    0.384 \\
    0.614 \\
    \end{bmatrix}
    $$

    After $1000$ steps, there is an almost zero probability that the fight is still going on.

    \hfill

    A transition matrix can be written in the following form

    $$
    T = \begin{bmatrix}
    Q & O_{r \times s} \\
    R & I_s
    \end{bmatrix}
    $$

    where $r$ represents the number of transient states and $s$ the number of absorbing states.
    The elements are described as follows

    \begin{itemize}
    \item Q is an $r \times r$ matrix that holds the probabilities of moving from a transient state to another transient state
    \item R is an $s \times r$ matrix that holds the probabilities of moving from a transient state to an absorbing state
    \item O is an $r \times s$ matrix that holds the probabilities of moving from an absorbing state to a transient state (which is impossible)
    \item I is an $s \times s$ matrix that holds the probabilities of moving between absorbing states (which is also impossible)
    \end{itemize}

    In the simplified fight, the elements are constructed as follows

    $$
    Q = \begin{bmatrix}
    0.6 & 0.75 & 0.95 & 0.6 & 0.85 \\
    0.3 & 0 & 0 & 0 & 0 \\
    0 & 0.25 & 0 & 0 & 0 \\
    0.1 & 0 & 0 & 0 & 0 \\
    0 & 0 & 0 & 0.4 & 0 \\
    \end{bmatrix}
    $$

    $$
    O_{r \times s} = \begin{bmatrix}
    0 & 0 \\
    0 & 0 \\
    0 & 0 \\
    0 & 0 \\
    0 & 0
    \end{bmatrix}
    $$


    $$
    R = \begin{bmatrix}
    0 & 0 & 0.05 & 0 & 0 \\
    0 & 0 & 0 & 0 & 0.15
    \end{bmatrix}
    $$

    $$
    I= \begin{bmatrix}
    1 & 0 \\
    0 & 1
    \end{bmatrix}
    $$

    The variable of main interest is  $\lim_{k\to\infty}T^Kx_0$.

    $$
    T^2 = \begin{bmatrix}
    Q & O_{r \times s} \\
    R & I_s
    \end{bmatrix}
    \begin{bmatrix}
    Q & O_{r \times s} \\
    R & I_s
    \end{bmatrix}
    =
    \begin{bmatrix}
    Q^2 & O_{r \times s} \\
    RQ +  R & I_s
    \end{bmatrix}
    $$

    $$
    T^3 = \begin{bmatrix}
    Q & O_{r \times s} \\
    R & I_s
    \end{bmatrix}
    \begin{bmatrix}
    Q^2 & O_{r \times s} \\
    RQ +  R & I_s
    \end{bmatrix}
    =
    \begin{bmatrix}
    Q^3 & O_{r \times s} \\
    RQ^2+RQ+R & I_s
    \end{bmatrix}
    $$

    In general, the pattern can be written as

    $$
    \begin{bmatrix}
    Q^k & O_{r \times s} \\
    R + RQ + ... + RQ^{k-1} & I_s
    \end{bmatrix}
    =
    \begin{bmatrix}
    Q^k & O_{r \times s} \\
    R\sum^{k-1}_{i=0}Q^i & I_s
    \end{bmatrix}
    $$

    By calculating the lower left element of the matrix, we encounter the series

    $$
    \sum_{i=0}^\infty Q^i = I + Q + Q^2 + Q^3 + ...
    $$

    This series converges to $(I-Q)^{-1}$ when some power $Q^k$ has columns sums that are less than 1.

    The above holds for the example fight and, therefore,

    $$
    (I-Q)^{-1} = \begin{bmatrix}
    102.56 & 101.28 & 97.43 & 96.41 & 87.18 \\
    30.77 & 31.38 & 29.23 & 28.92 & 26.15 \\
    7.69 & 7.85 & 8.31 & 7.23 & 6.54 \\
    10.26 & 10.13 & 9.74 & 10.64 & 8.72 \\
    4.10 & 4.05 & 3.90 & 4.26 & 4.49
    \end{bmatrix}
    $$

    and

    $$
    R(I-Q)^{-1} = \begin{bmatrix}
    0.38 & 0.39 & 0.42 & 0.36 & 0.33 \\
    0.62 & 0.61 & 0.58 & 0.64 & 0.67
    \end{bmatrix}
    $$

    Consequently,


    $$
    \lim_{k\to\infty}T^K = \begin{bmatrix}
    0 & 0 & 0 & 0 & 0 & 0 & 0 \\
    0 & 0 & 0 & 0 & 0 & 0 & 0 \\
    0 & 0 & 0 & 0 & 0 & 0 & 0 \\
    0 & 0 & 0 & 0 & 0 & 0 & 0 \\
    0 & 0 & 0 & 0 & 0 & 0 & 0 \\
    0.38 & 0.39 & 0.42 & 0.36 & 0.33 & 1 & 0\\
    0.62 & 0.61 & 0.58 & 0.64 & 0.67 & 0 & 1
    \end{bmatrix}
    $$

    A fight always starts with both fighters standing across each other.
    To find the probabilities of each state after "a very long time", we compute

    $$
    \lim_{k\to\infty}T^Kx_0 = \begin{bmatrix}
    0 & 0 & 0 & 0 & 0 & 0 & 0 \\
    0 & 0 & 0 & 0 & 0 & 0 & 0 \\
    0 & 0 & 0 & 0 & 0 & 0 & 0 \\
    0 & 0 & 0 & 0 & 0 & 0 & 0 \\
    0 & 0 & 0 & 0 & 0 & 0 & 0 \\
    0.38 & 0.39 & 0.42 & 0.36 & 0.33 & 1 & 0\\
    0.62 & 0.61 & 0.58 & 0.64 & 0.67 & 0 & 1
    \end{bmatrix}
    \begin{bmatrix}
    1 \\
    0 \\
    0 \\
    0 \\
    0 \\
    0 \\
    0
    \end{bmatrix}
    =
    \begin{bmatrix}
    0 \\
    0 \\
    0 \\
    0 \\
    0 \\
    0.38 \\
    0.62
    \end{bmatrix}
    $$

    For an absorbing Markov chain, the $(i, j)$-entry of $R(I_r-Q)^{-1}$ is the probability of ending in absorbing state $a_i$ given that we started in transient state $t_j$.
    In addition, the $(i, j)$-entry of $(I_r-Q)^{-1}$ contains the expected number of visits to transient state $t_i$ given that we started in transient state $t_j$.
    With the help of these properties it is possible to determine the probabilities of different outcomes and forecast the number of actions in a fight.

    \begin{center}
        \begin{tabular}{l| c c c}
        \textbf{Variable of interest} & \textbf{blue} & \textbf{red} \\ [0.5ex]
        \hline
        Probability of knockout & 38\% & 62\% \\
        Number of strikes attempted & 30.77 & 10.26 \\
        Number of strikes landed  & 7.69 & 4.10 \\
        \end{tabular}
    \end{center}

\end{document}