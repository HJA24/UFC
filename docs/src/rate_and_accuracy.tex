%! Author = huibmeulenbelt
%! Date = 18/04/2025

% Preamble
\documentclass[11pt]{article}

% Packages
\usepackage{amsmath}

% Document
\begin{document}
    \title{Work-rate and accuracy models}
    \date{}
    \maketitle

    \section*{Work-rate}
    The number of strikes, takedowns, submission attempts and other actions are referred to as the 'work-rates'.
    While these "rates" may seem fighter-specific, the ability to prevent one’s opponent from attempting techniques is a crucial skill in itself.
    This can be through range control or rendering the opponent unable to attempt techniques through grappling.
    Consequently, we allow for an attack and defence parameter in each of these models.
    With the help of an appropriate distribution and the inferred parameters, it is possible to predict the number of actions in a round.

    The Poisson distribution is a discrete distribution that calculates the probability of a certain number of events occurring within a fixed interval.
    It has only one parameter, $\lambda$, that is equal to both the mean and the variance.

    $$
    \text{SA} \sim Poisson(\gamma_{str} + \lambda_{attack, blue} - \lambda_{defence, red})
    $$

    \section*{Accuracy}
    Next, it is necessary to know the accuracy of each strike, takedown and other possible actions.
    Again, we allow for an attack and defence parameter in each of the accuracy models.

    The binomial distribution is a discrete distribution that calculates the probability of each of the possible number of successes on

    $$
    \text{SL} \sim Binomial_logit(\text{SA}, \gamma_{sta} + \lambda_{attack, blue} - \lambda_{defence, red})
    $$

     logit


\end{document}