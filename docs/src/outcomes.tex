%! Author = huibmeulenbelt
%! Date = 11/04/2025

% Preamble
\documentclass[11pt]{article}

% Packages
\usepackage{amssymb}
\usepackage{amsmath}
\usepackage{adjustbox}

\pagestyle{empty}

% Document
\begin{document}
    \title{Fight outcomes}
    \author{HJA Meulenbelt}
    \maketitle

    \section*{Probabilities}
    Each fight consists of $f \in \{blue, red\}$ fighters and is observed by $J=3$ judges.

    Table \ref{tab:table} shows the possible decision outcomes of fight as a result of the judges' verdicts.

    \begin{table}[!ht]
        \caption{Overview of all possible decisions}
        \vspace{0.2cm}
        \centering
        \begin{tabular}{l l l l l}
            \textbf{Judge 1} & \textbf{Judge 2} & \textbf{Judge 3} & \textbf{Winner} & \textbf{Decision} \\
            \hline
            blue & blue & blue & blue & unanimous win \\
            blue & blue & red & blue & split win \\
            blue & blue & draw & blue & majority win \\
            blue & red & draw & draw & split draw \\
            blue & draw & draw & draw & majority draw \\
            draw & draw & draw & draw & unanimous draw
        \end{tabular}\label{tab:table}
    \end{table}

    Due to their rare nature, draws are excluded from the possible judges' verdicts.
    If a fight goes to the distance, judge $j$ assigns per round $r$ a score $y^{a, b}_{j, r} \in \{7-10, 8-10, 9-10, 10-9, 10-8, 10-7\}$.
    A fight goes to the distance if one of the fighters is not prematurely defeated by either a knockout or a submission.
    The probability of a knockout and a submission during a round is defined as, respectively, $p_{knockout, r, f}$ and $p_{submission, r, f}$.
    It is assumed that these probabilities are constant and do not change as the fight progresses.
    Therefore, the probability of "surviving" any round is equal to
    $$
    p_{round} = 1 - (p_{knockout, blue} + p_{knockout, red} + p_{submission, blue} + p_{submission, red})
    $$

    The main event and co-main event are scheduled for $R=5$ rounds.
    All the other events are scheduled for $R=3$ rounds.

    Logically, the probability of a fight "going the distance" is
    $$
    p_{decision} =\begin{cases}
                      p_{round}^5 & \text{if (co-)main fight} \\
                      p_{round}^3 & \text{other}
    \end{cases}
    $$

    In this case a fighter can win by either an unanimous or a split decision.

    The probability of fighter $f$ winning by unanimous decision is equal to
    $$
    p_{unanimous, f} = p_{decision} \cdot \sum_u^U
    $$

    and by split decision
    $$
    p_{split, f} = p_{decision} \cdot \sum_{j=1}^J\mathbb{I}(a>b) = J - 1
    $$

    Together these probabilities form the probability that fighter $f$ wins by the judges' decisions.
    $$
    p_{judges, f} = p_{unanimous, f} + p_{split, f}
    $$

    The probabilities of the other methods to win the fight are
    $$
    p_{submission, f} = p_{submission, r, f} + \sum_{r=1}^{R-1} rp_{round} \cdot p_{submission, r, f}
    $$

    and
    $$
    p_{knockout, f} = p_{knockout, r, f} + \sum_{r=1}^{R-1} rp_{round} \cdot p_{knockout, r, f}
    $$

    Together, the probability of fighter $f$ winning the fight is equal to
    $$
    p_{win, f} = p_{knockout, f} +  p_{submission, f} + p_{judges, f}
    $$

\end{document}