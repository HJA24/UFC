%! Author = huibmeulenbelt
%! Date = 18/01/2026

% Preamble
\documentclass[11pt]{article}

% Packages
\usepackage{amsmath}

% Document
\begin{document}
    \title{Network properties}
    \maketitle


    \section*{Clustering}
    Clustering measures how often a fighter’s opponents have also fought each other.

    \[
        C_i = \frac{2e_i}{k_i(k_i-1)},
        \qquad
        C = \frac{1}{|V|}\sum_i C_i
    \]

    \textbf{Interpretation.}
    \(C=0\): opponents of a fighter never fight each other, producing tree-like local structure.
    \(C=1\): every fighter’s opponents form a clique, yielding maximal local redundancy.

    \textbf{Relation to Bayesian inference.}
    High clustering provides multiple independent comparison paths between fighters, reducing posterior uncertainty and enabling consistency checks for transitive skill differences.

    \section*{Transitivity}
    Transitivity measures the global prevalence of triangular fight relationships.

    \textbf{Formula.}
    \[
        T = \frac{3 \times \text{number of triangles}}{\text{number of connected triples}}
    \]

    \textbf{Interpretation.}
    \(T=0\): no closed triplets exist; transitive comparisons rely on long chains.
    \(T=1\): all connected triples are closed; comparisons are maximally redundant.

    \textbf{Relation to Bayesian inference.}
    High transitivity increases the number of overlapping constraints on relative skill, stabilizing posterior estimates of win probabilities for unobserved matchups.

    \section*{Efficiency}
    Effiency measures how short fight-path distances are across the network.

    \textbf{Formula}
    \[
        E = \frac{1}{n(n-1)} \sum_{i\neq j} \frac{1}{d_{ij}}
    \]

    \textbf{Interpretation}
    \(E=0\): many fighter pairs are unreachable or separated by long paths.
    \(E=1\): all fighters are directly connected or separated by a single fight.

    \textbf{Relation to Bayesian inference}
    Higher efficiency means fewer intermediaries (fights between opponents...) are needed to relate fighters, limiting uncertainty accumulation when inferring transitive win probabilities.

    \section*{Connectivity}
    Connectivity measures whether all fighters belong to a single connected component.

    \textbf{Formula}
    \[
        \kappa = \frac{|\text{largest connected component}|}{|V|}
    \]

    \textbf{Interpretation}
    \(\kappa=0\): the network is entirely fragmented into isolated nodes.
    \(\kappa=1\): the fight graph is fully connected.

    \textbf{Relation to Bayesian inference}
    Only connected components allow data-driven comparisons; disconnected components force cross-group win probabilities to be driven by prior assumptions rather than evidence.

    \section*{Closeness}
    \section*{Degree centrality}
    The degree centrality for a node $v$ is the fraction of nodes it is connected to.
    It is normalized by dividing by the maximum possible degree in a simple graph, $n - 1$ where $n$ is the number of nodes in $G$.

\end{document}